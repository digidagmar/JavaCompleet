\documentclass[11pt,letterpaper]{article}
%\topmargin -.45in
\textwidth 6.5in
\textheight 9.in
\oddsidemargin 0in
\headheight 0in
\usepackage{graphicx}
\usepackage{fancybox}
%\usepackage{palatino}
\usepackage[utf8]{inputenc} %solucion del problema de los acentos.
\usepackage{epsfig,graphicx}
\usepackage{multicol,pst-plot}
\usepackage{pstricks}
\usepackage{amsmath}
\usepackage{amsfonts}
\usepackage{amssymb}
\usepackage{eucal}
\usepackage[left=2cm,right=2cm,top=2cm,bottom=2cm]{geometry}
\pagestyle{empty}
\DeclareMathOperator{\tr}{Tr}
\newcommand*{\op}[1]{\check{\mathbf#1}}
\newcommand{\bra}[1]{\langle #1 |}
\newcommand{\ket}[1]{| #1 \rangle}
\newcommand{\braket}[2]{\langle #1 | #2 \rangle}
\newcommand{\mean}[1]{\langle #1 \rangle}
\newcommand{\opvec}[1]{\check{\vec #1}}
\renewcommand{\sp}[1]{$${\begin{split}#1\end{split}}$$}

\usepackage{lipsum}

\usepackage{listings}
\usepackage{color}

\definecolor{codegreen}{rgb}{0,0.6,0}
\definecolor{codegray}{rgb}{0.5,0.5,0.5}
\definecolor{codepurple}{rgb}{0.58,0,0.82}
\definecolor{backcolour}{rgb}{0.95,0.95,0.92}

\lstdefinestyle{mystyle}{
	backgroundcolor=\color{backcolour},   
	commentstyle=\color{codegreen},
	keywordstyle=\color{magenta},
	numberstyle=\tiny\color{codegray},
	stringstyle=\color{codepurple},
	basicstyle=\footnotesize,
	breakatwhitespace=false,         
	breaklines=true,                 
	captionpos=b,                    
	keepspaces=true,                 
	numbers=left,                    
	numbersep=5pt,                  
	showspaces=false,                
	showstringspaces=false,
	showtabs=false,                  
	tabsize=2
}

\lstset{style=mystyle}

\begin{document}
	\pagestyle{plain}
	\begin{flushleft}
		Code Gorilla \\
		
		\smallskip
		
		Hoofdstukken: \\
		
\hspace{1cm} 8. Overerving \\
\hspace{1cm} 9. Abstracte klassen en interfaces \\
		
		Onderwerp: Design en implementatie \\
		
	\end{flushleft}
	
	\begin{flushright}\vspace{-5mm}
		\includegraphics[height=1.5cm]{gorilla.png}
	\end{flushright}
	
	\begin{center}\vspace{-1cm}
		\textbf{ \large Java Opdracht}\\
		
	\end{center}
	
	
	\rule{\linewidth}{0.1mm}
	%%%%%%%%%%%%%%%%%%%%%%%%%%%%%%%%%%%%%%%%%%%%%%%%%%%%%%%%%%%%%%%%%%%%%%%%
	
	\bigskip
	\textbf{\large{Opdracht omschrijving}}
	\\
	
	\bigskip
	
	De koffiezetter: \\
	\begin{enumerate}
		\item Maak een domeinmodel voor een koffiezetter
		\item De koffiezetter heeft:
		\begin{itemize}
			\item Hoeveelheid suiker
			\item Hoeveelheid melk
			\item Cappuccino, Gewone koffie, en Chocolademelk
			\item Een prijs (bv. Chocolademelk 1.20, koffie 1.50 etc. )
			\item Functies voor aan/uit zetten selecteren etc
		\end{itemize}
		\item Maak het klassendiagram, gebruik interfaces voor Item(koffie, chocolade, cappuccino)
		\item Implementeer de applicatie
	\end{enumerate}

	\rule{\linewidth}{0.1mm}

	\textbf{Voorbeeld van input/output}
	
	\bigskip
	
\texttt{
> status \\
\indent \indent Koffiezetter staat uit \\
\indent $ $ > zet aan \\
\indent \indent Koffiezetter is aangezet \\
\indent $ $ > status \\
\indent \indent Koffiezetter staat aan \\
\indent $ $ > select cappucino \\
\indent $ $ > toon selectie \\
\indent \indent Selectie: cappucino \\
\indent $ $ > select koffie \\
\indent $ $ > toon selectie \\
\indent \indent Selectie: koffie \\
\indent $ $ > toon prijzen \\
\indent \indent koffie: 1.35 \\
\indent \indent cappuccino: 1.85 \\
\indent \indent chocolade: 2.35 \\
}

	
\end{document}
