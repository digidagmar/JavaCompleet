\documentclass[11pt,letterpaper]{article}
%\topmargin -.45in
\textwidth 6.5in
\textheight 9.in
\oddsidemargin 0in
\headheight 0in
\usepackage{graphicx}
\usepackage{fancybox}
%\usepackage{palatino}
\usepackage[utf8]{inputenc} %solucion del problema de los acentos.
\usepackage{epsfig,graphicx}
\usepackage{multicol,pst-plot}
\usepackage{pstricks}
\usepackage{amsmath}
\usepackage{amsfonts}
\usepackage{amssymb}
\usepackage{eucal}
\usepackage[left=2cm,right=2cm,top=2cm,bottom=2cm]{geometry}
\pagestyle{empty}
\DeclareMathOperator{\tr}{Tr}
\newcommand*{\op}[1]{\check{\mathbf#1}}
\newcommand{\bra}[1]{\langle #1 |}
\newcommand{\ket}[1]{| #1 \rangle}
\newcommand{\braket}[2]{\langle #1 | #2 \rangle}
\newcommand{\mean}[1]{\langle #1 \rangle}
\newcommand{\opvec}[1]{\check{\vec #1}}
\renewcommand{\sp}[1]{$${\begin{split}#1\end{split}}$$}

\usepackage{lipsum}

\usepackage{listings}
\usepackage{color}

\definecolor{codegreen}{rgb}{0,0.6,0}
\definecolor{codegray}{rgb}{0.5,0.5,0.5}
\definecolor{codepurple}{rgb}{0.58,0,0.82}
\definecolor{backcolour}{rgb}{0.95,0.95,0.92}

\lstdefinestyle{mystyle}{
	backgroundcolor=\color{backcolour},   
	commentstyle=\color{codegreen},
	keywordstyle=\color{magenta},
	numberstyle=\tiny\color{codegray},
	stringstyle=\color{codepurple},
	basicstyle=\footnotesize,
	breakatwhitespace=false,         
	breaklines=true,                 
	captionpos=b,                    
	keepspaces=true,                 
	numbers=left,                    
	numbersep=5pt,                  
	showspaces=false,                
	showstringspaces=false,
	showtabs=false,                  
	tabsize=2
}

\lstset{style=mystyle}

\begin{document}
	\pagestyle{plain}
	\begin{flushleft}
		Code Gorilla \\
		
		\smallskip
		
		Hoofdstuk: 1. Basis en Algoritmiek \\
		Onderwerp: Arrays \\
		
	\end{flushleft}
	
	\begin{flushright}\vspace{-5mm}
		\includegraphics[height=1.5cm]{gorilla.png}
	\end{flushright}
	
	\begin{center}\vspace{-1cm}
		\textbf{ \large Java Opdracht 1 }\\
		
	\end{center}
	
	
	\rule{\linewidth}{0.1mm}
	%%%%%%%%%%%%%%%%%%%%%%%%%%%%%%%%%%%%%%%%%%%%%%%%%%%%%%%%%%%%%%%%%%%%%%%%
	
	\bigskip
	\textbf{\large{Opdracht omschrijving}}
	\\
	
	\smallskip
	
	Maak een Java applicatie met de volgende functionaliteiten: \\
	\begin{enumerate}
		\item Die een klasse 'Dobbelsteen' bevat.
		\item In een \textbf{int} array 5 dobbelsteen waarden kan opslaan.
		\item Die een attribuut bevat waar de waarden van de laaste worp wordt opgeslagen.
		\item Die een methode bevat die een van de volgende String waarden teruggeeft: 
		\begin{itemize}
			\item 'Three of a Kind'
			\item 'Four of a Kind'
			\item 'Small Street'
			\item 'Large Street'
			\item 'Full House'
			\item 'Chance'
			\item 'Yahtzee'
		\end{itemize}
	\item Die een methode bevat die de hoeveelheid punten van de laaste worp teruggeeft.
	\end{enumerate}
	
	\rule{\linewidth}{0.1mm}

	\bigskip
	\textbf{\large{Voorbeeld van output}} \\
	
	\smallskip
	
	Worp 1: 2 3 6 5 3 \\ 
\indent	Type: Chance  \\ 
\indent	Punten: 19 \\	
		
	Worp 2: 3 3 4 4 4  \\ 
\indent	Type: Full House  \\	
\indent	Punten: 18  \\	
	
	Worp 3: 5 5 6 3 5  \\ 
\indent	Type: Three of a Kind  \\	
\indent	Punten: 24  \\	
	
	etc. \\ 
	
\end{document}
